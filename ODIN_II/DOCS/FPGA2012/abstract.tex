As the semiconductor industry is driven towards large scale heterogeneous chips,
it is likely that one section of these systems may well be an FPGA.  To facilitate
the development of appropriate FPGA architectures and CAD tools for those systems,
as well as mainstream pure FPGAs, there is a need for a large scale, publically available,
software suite that can synthesize, into hypothetical human-described architectures, 
from the HDL level through logical and physical synthesis, to the detailed modeling
of area, performance and energy.  This paper describes such a flow, in the 'Verilog to Routing'
(VTR) project, which is a broad world-wide collaboration.   There are three core tools: ODIN II
for Verilog Elaboration and front-end hard-block synthesis, ABC for logic synthesis, and VPR for
physical synthesis and analysis.  In this paper we describe the current state of this complete
flow. ODIN II now has a simulation capability to help verify that the flow output is correct,
at all levels [THAT's not true, is it], as well as specialized synthesis at the elaboration step
for multipliers and memories. We make use ABC's ability to look inside hard block's timing
to optimize soft logic.  There is now has a timing-driven (VPR) back-end that can target far more complex logic blocks
than in the past.  Equally important, we present a new set of benchmark circuits that are larger and
more complex than those used in the past.  Finally, we illustrate the use of the new
flow by using it to show how it can help architect a floating-point unit in an FPGA.

