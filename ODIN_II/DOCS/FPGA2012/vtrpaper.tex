% This is "sig-alternate.tex" V1.3 OCTOBER 2002
% This file should be compiled with V1.6 of "sig-alternate.cls" OCTOBER 2002
%
% This example file demonstrates the use of the 'sig-alternate.cls'
% V1.6 LaTeX2e document class file. It is for those submitting
% articles to ACM Conference Proceedings WHO DO NOT WISH TO
% STRICTLY ADHERE TO THE SIGS (PUBS-BOARD-ENDORSED) STYLE.
% The 'sig-alternate.cls' file will produce a similar-looking,
% albeit, 'tighter' paper resulting in, invariably, fewer pages.
%
% ----------------------------------------------------------------------------------------------------------------
% This .tex file (and associated .cls V1.6) produces:
%       1) The Permission Statement
%       2) The Conference (location) Info information
%       3) The Copyright Line with ACM data
%       4) NO page numbers
%
% as against the acm_proc_article-sp.cls file which
% DOES NOT produce 1) thru' 3) above.
%
% Using 'sig-alternate.cls' you have control, however, from within
% the source .tex file, over both the CopyrightYear
% (defaulted to 2002) and the ACM Copyright Data
% (defaulted to X-XXXXX-XX-X/XX/XX).
% e.g.
% \CopyrightYear{2003} will cause 2002 to appear in the copyright line.
% \crdata{0-12345-67-8/90/12} will cause 0-12345-67-8/90/12 to appear in the copyright line.
%
% ---------------------------------------------------------------------------------------------------------------
% This .tex source is an example which *does* use
% the .bib file (from which the .bbl file % is produced).
% REMEMBER HOWEVER: After having produced the .bbl file,
% and prior to final submission, you *NEED* to 'insert'
% your .bbl file into your source .tex file so as to provide
% ONE 'self-contained' source file.
%
% ================= IF YOU HAVE QUESTIONS =======================
% Questions regarding the SIGS styles, SIGS policies and
% procedures, Conferences etc. should be sent to
% Adrienne Griscti (griscti@acm.org)
%
% Technical questions _only_ to
% Gerald Murray (murray@acm.org)
% ===============================================================
%
% For tracking purposes - this is V1.3 - OCTOBER 2002

\documentclass{sig-alternate}
%\documentclass{sigplan-proc}
\usepackage{amsmath,amssymb}                        % better math environment
\usepackage{booktabs}                       % better tables
\usepackage{sepnum}                         % fancy number formating
%\usepackage{epigraph}                       % quotes at beginning of chapters
\usepackage[all]{xy}                        % drawing graphs
\usepackage[binary,cdot,amssymb]{SIunits}   % automatic units
\usepackage{url}                            % typeset urls
%\usepackage{setspace}                       % enable double and half spacing
%%\usepackage[toc,page]{appendix}             % makes appendices nicer
%%\usepackage[dvips]{graphicx}                % include eps/png graphics
%\usepackage{epsfig}
\usepackage{graphicx}
%\usepackage[hang]{subfigure}                % allow subfigures
\usepackage{acronym}                        % table of acronyms
%\usepackage{paralist}                       % compact lists
%\usepackage{listings}
\usepackage{xspace}                         % Make my macro's easy to use
\usepackage{multirow}
\usepackage[hyperref,colorlinks=false,ps2pdf,pdfborder={0 0 0}]{hyperref}  % make links when ps2pdf is used.  if we want pdflatex use \usepackage[pdftex,colorlinks]{hyperref} instead
\hypersetup{
pdfauthor = {Author List},
pdftitle = {The VTR Project: Architecture and CAD for FPGAs from Verilog to Routing}}


\newcommand{\nicenum}[1]{\sepnum{.}{\thinspace}{\thinspace}{#1}}
\newcommand{\niceunit}[2]{\unit{\nicenum{#1}}{#2}}
\newcommand{\percent}{\%}
\renewcommand{\chapterautorefname}{Chapter}
\renewcommand{\figureautorefname}{Fig.}
\renewcommand{\sectionautorefname}{Section}
\renewcommand{\subsectionautorefname}{Section}
\renewcommand{\subsubsectionautorefname}{Section}
\newcommand{\etal}{\emph{et al.}\xspace}

\clubpenalty=10000
\widowpenalty=10000

%Who cares about pretty just make everything fit
%\renewcommand\floatpagefraction{.9}
%\renewcommand\topfraction{.9}
%\renewcommand\bottomfraction{.9}
%\renewcommand\textfraction{.1}

%IMPORTANT NOTE - comment out next two lines to remove numbering for the final copy
%\pagenumbering{arabic}
%\pagestyle{plain}
\hyphenation{prog-ram-mable}
\hyphenation{opt-imi-zed}
\hyphenation{evo-lve}

\begin{document} 

% Squeeze the lines a bit to give us that extra bit of space, keep near 1
\renewcommand{\baselinestretch}{0.968}
%
% --- Author Metadata here ---
%\conferenceinfo{FPGA'12,} {February XX--March YY, 2012, Monterey, California, USA.} 
%\CopyrightYear{2012} 
%\crdata{978-1-4503-0554-9/11/02} 
%\clubpenalty=10000 
%\widowpenalty = 10000
% --- End of Author Metadata ---

\title{The VTR Project: Architecture and CAD for FPGAs from Verilog to Routing}
%
% You need the command \numberofauthors to handle the "boxing"
% and alignment of the authors under the title, and to add
% a section for authors number 4 through n.
%
% Up to the first three authors are aligned under the title;
% use the \alignauthor commands below to handle those names
% and affiliations. Add names, affiliations, addresses for
% additional authors as the argument to \additionalauthors;
% these will be set for you without further effort on your
% part as the last section in the body of your article BEFORE
% References or any Appendices.


\numberofauthors{1}
%
% You can go ahead and credit authors number 4+ here;
% their names will appear in a section called
% "Additional Authors" just before the Appendices
% (if there are any) or Bibliography (if there
% aren't)

% Put no more than the first THREE authors in the \author command
\author{
%
% The command \alignauthor (no curly braces needed) should
% precede each author name, affiliation/snail-mail address and
% e-mail address. Additionally, tag each line of
% affiliation/address with \affaddr, and tag the
%% e-mail address with \email.
\alignauthor Authors\\
       \affaddr{Department of Electrical and Computer Engineering}\\
       \affaddr{University}\\
       \email{A|B|C@university.edu}
}
       %\affaddr{Address}\\
\date{date}
\maketitle
\begin{abstract}
As the semiconductor industry is driven towards large scale heterogeneous chips,
it is likely that one section of these systems may well be an FPGA.  To facilitate
the development of appropriate FPGA architectures and CAD tools for those systems,
as well as mainstream pure FPGAs, there is a need for a large scale, publically available,
software suite that can synthesize, into hypothetical human-described architectures, 
from the HDL level through logical and physical synthesis, to the detailed modeling
of area, performance and energy.  This paper describes such a flow, in the 'Verilog to Routing'
(VTR) project, which is a broad world-wide collaboration.   There are three core tools: ODIN II
for Verilog Elaboration and front-end hard-block synthesis, ABC for logic synthesis, and VPR for
physical synthesis and analysis.  In this paper we describe the current state of this complete
flow. ODIN II now has a simulation capability to help verify that the flow output is correct,
at all levels [THAT's not true, is it], as well as specialized synthesis at the elaboration step
for multipliers and memories. We make use ABC's ability to look inside hard block's timing
to optimize soft logic.  There is now has a timing-driven (VPR) back-end that can target far more complex logic blocks
than in the past.  Equally important, we present a new set of benchmark circuits that are larger and
more complex than those used in the past.  Finally, we illustrate the use of the new
flow by using it to show how it can help architect a floating-point unit in an FPGA.


\end{abstract}

% The body should reexpand any acronyms
\acresetall

% A category with the (minimum) three required fields
\category{B.5.2}{Design Aids}{Automatic Synthesis, Optimization}
\terms{Algorithms, Design, Architecture, Measurement, Performance}
%A category including the fourth, optional field follows...



%%%%%%%%%%%%%%%%%%%%%%%%%%%%%%%%%%%%%%%%%%%%%%%%%%%%%%%%%%%%%%%%%%%%%%%%%%%%%%
\section{Introduction}

The exploration of new programmable architectures, and the development of innovative algorithms involved
in their commercial requires a robust CAD and architectural software synthesis flow that permits experimentation.
In order to model modern and future architectures, such a software flow is necessarily quite complex, and largely beyond the
capacity of any single academic enterprise to create, evolve and maintain.  Equivalent commercial flows are
supported by hundreds of full-time engineers.  Equally important to serve the same needs is a set of relevant
large-scale circuit benchmarks that can be used to test architectures and algorithms.
This paper describes the status of a global collaboration attempting to provide such a framework, describing
several innovations with the three main parts of the tool flow, new work to create robust benchmarks, and 
an illustration of the flow's ability to explore a new kind of hard logic block.

One of the goals of this project is to create a place where enhancements by others can be more easily integrated
into the suite of tools, rather than being orphaned, as is often the case.  This can be challenging as making
a flow robust requires more work than a typical academic project and publication requires.

This paper is organized as follows:  the next section gives an overview of the entire, flow describing its various inputs
and outputs.  The three subsequent sections describe the individual portions of the flow, and the new features just
becoming available.  In Section 6 we describe the work to develop larger, modern benchmarks, and describe the currently
available suite, and some data making it through the flow.  Section 7 provides a case study of the use of the flow to replicate previous work (done on an branch
of the VPR\cite{betz_vpr99}) to model floating-point logic blocks.  Section 8 outlines the extensive set of future work we see as necessary
to continue this work.

%%%%%%%%%%%%%%%%%%%%%%%%%%%%%%%%%%%%%%%%%%%%%%%%%%%%%%%%%%%%%%%%%%%%%%%%%%%%%%

%%%%%%%%%%%%%%%%%%%%%%%%%%%%%%%%%%%%%%%%%%%%%%%%%%%%%%%%%%%%%%%%%%%%%%%%%%%%%%
\section{Overview}
\label{sec:overview}

In this section we describe the entire flow, from input to output.


\subsection{Inputs}
\subsection{Outputs}
\subsection{Models}


%%%%%%%%%%%%%%%%%%%%%%%%%%%%%%%%%%%%%%%%%%%%%%%%%%%%%%%%%%%%%%%%%%%%%%%%%%%%%%

%%%%%%%%%%%%%%%%%%%%%%%%%%%%%%%%%%%%%%%%%%%%%%%%%%%%%%%%%%%%%%%%%%%%%%%%%%%%%%
\section{Odin II: Elaboration}
\label{sec:odin2}

In this section, 

%%%%%%%%%%%%%%%%%%%%%%%%%%%%%%%%%%%%%%%%%%%%%%%%%%%%%%%%%%%%%%%%%%%%%%%%%%%%%%

%%%%%%%%%%%%%%%%%%%%%%%%%%%%%%%%%%%%%%%%%%%%%%%%%%%%%%%%%%%%%%%%%%%%%%%%%%%%%%
\section{ABC: Logic Synthesis}
\label{sec:abc}

In this section, 

%%%%%%%%%%%%%%%%%%%%%%%%%%%%%%%%%%%%%%%%%%%%%%%%%%%%%%%%%%%%%%%%%%%%%%%%%%%%%%

%%%%%%%%%%%%%%%%%%%%%%%%%%%%%%%%%%%%%%%%%%%%%%%%%%%%%%%%%%%%%%%%%%%%%%%%%%%%%%
\section{VPR: Physical Synthesis}
\label{sec:vpr}

In this section, 

%%%%%%%%%%%%%%%%%%%%%%%%%%%%%%%%%%%%%%%%%%%%%%%%%%%%%%%%%%%%%%%%%%%%%%%%%%%%%%

%%%%%%%%%%%%%%%%%%%%%%%%%%%%%%%%%%%%%%%%%%%%%%%%%%%%%%%%%%%%%%%%%%%%%%%%%%%%%%
\section{Odin II: Elaboration}
\label{sec:odin2}

In this section, 

%%%%%%%%%%%%%%%%%%%%%%%%%%%%%%%%%%%%%%%%%%%%%%%%%%%%%%%%%%%%%%%%%%%%%%%%%%%%%%

%%%%%%%%%%%%%%%%%%%%%%%%%%%%%%%%%%%%%%%%%%%%%%%%%%%%%%%%%%%%%%%%%%%%%%%%%%%%%%
\section{Example Use of Flow: Floating-Point Blocks}
\label{sec:exampleuse}

In this section, 

%%%%%%%%%%%%%%%%%%%%%%%%%%%%%%%%%%%%%%%%%%%%%%%%%%%%%%%%%%%%%%%%%%%%%%%%%%%%%%

%%%%%%%%%%%%%%%%%%%%%%%%%%%%%%%%%%%%%%%%%%%%%%%%%%%%%%%%%%%%%%%%%%%%%%%%%%%%%%
\section{Future Work}
\label{sec:futurework}


In this section, ....

%%%%%%%%%%%%%%%%%%%%%%%%%%%%%%%%%%%%%%%%%%%%%%%%%%%%%%%%%%%%%%%%%%%%%%%%%%%%%%

%%%%%%%%%%%%%%%%%%%%%%%%%%%%%%%%%%%%%%%%%%%%%%%%%%%%%%%%%%%%%%%%%%%%%%%%%%%%%%
\section{Conclusions}
\label{sec:conclusions}

We have presented ...

%%%%%%%%%%%%%%%%%%%%%%%%%%%%%%%%%%%%%%%%%%%%%%%%%%%%%%%%%%%%%%%%%%%%%%%%%%%%%%


\acrodef{FPGA}{Field-programmable gate array}
\acrodef{CAD}{computer-aided design}
\acrodef{ASIC}{application-specific integrated circuit}
\acrodef{NRE}{non-recurring engineering}
\acrodef{HDL}{hardware description language}
\acrodef{VHDL}{VHSIC Hardware Description Language}
\acrodef{ROM}{Read-only Memory}
\acrodef{RAM}{Random-access Memory}
\acrodef{QIS}{Quartus II Integrated Synthesis}
\acrodef{DSP}{Digital Signal Processing}
\acrodef{LUT}{lookup table}
\acrodef{MPGA}{Mask-programmable Gate Array}
\acrodef{IC}{Integrated Circuit}
\acrodef{RTL}{Register Transfer Level}
\acrodef{VCD}{Value Change Dump}
\acrodef{BLE}{Basic Logic Element}
\acrodef{CMP}{Circuits Multi-Projets}
\acrodef{DFT}{Design for Testability}
\acrodef{LAB}{Logic Array Block}
\acrodef{ALM}{Adaptive Logic Module}
\acrodef{CLB}{Cluster-based Logic Block}
\acrodef{ITRS}{International Technology Roadmap for Semiconductors}


\scriptsize

\bibliographystyle{abbrv}%Remove nat if not using natbib

\bibliography{references}


\balancecolumns % GM July 2000

% That's all folks!
\end{document}
